\documentclass[12pt]{article}

\usepackage{sbc-template}

\usepackage{graphicx,url}

%\usepackage[brazil]{babel}   
\usepackage[latin1]{inputenc}  

     
\sloppy

\title{PlantGoshi\\ Projeto Integrador III - Sistema Aut\^onomo}

\author{Anderson J. Silva, Felipe R. de Luca, Nelson J. Dressler }


\address{Bacharelado em Ci\^encia da Computa\c c\~ao -- Centro Universit\'ario Senac - Santo Amaro \\
  S\~ao Paulo - SP - Brasil \\ 2015
}
\begin{document} 

\maketitle
     
\begin{resumo} 
 
 O projeto consiste em um jogo digital em 2D, desenvolvido em linguagem C, onde o jogador deve cuidar de uma
 \'arvore em seu processo de crescimento, com o objetivo principal de colher os melhores frutos.
 Para isso, o jogador ter\'a
 como ferramenta de intera\c c\~ao uma varinha m\'agica, que permitir\'a aplicar poderes
 que interajam com os elementos dentro do jogo, contribuindo
 com o crescimento da \'arvore e impedindo que pragas ataquem os frutos.  
 
 A varinha consiste em um LED RGB conectado a uma placa controladora (Arduino) e poder\'a
 assumir uma quantidade de cores limitada, correspondendo aos poderes presentes no jogo.
 
 A intera\c c\~ao da varinha com o jogo ser\'a por interm\'edio do reconhecimento do LED nas imagens capturadas pela
 c\^amera de video instalada no computador, processadas por algoritmos baseados em levantamento bibliogr\'afico.
 
\end{resumo}

\end{document}
